\section{Conclusion}

We noticed that the correlation between complexity and errors is not as expected. In general, a larger file with more complexity should have more bugs, but we found that the the smaller files had almost as many bugs as the largest files. Moreover, files with average complexity had the least number of bugs. We attribute this to there being less code in smaller files, so less error checking. Thus, the medium sized files have similar logic to files of lower complexity, but spend more lines of code to catch errors. \\

Moreover, after mutating the scripts, the number of bugs that Oyente found increased nearly 5 times, where Mythril and SmartCheck found nearly the same number of bugs on average. This can likely be attributed to the types of bugs that Mythril searches for. With chain execution, Mythril misses bugs related to bad operations as the paths remain relatively unchanged. Interestingly, Mythril found many more issues in the mutated scripts, providing credence to the idea that Mythril treats mutations differently from Oyente or SmartCheck. In conclusion, we find that the Oyente testing suite significantly more bugs than Mythril on most scripts, and more bugs than SmartCheck on complex scripts. In contrast, the Mythril tool provides more details of each bugs, and the SmartCheck focuses on vastly different bugs altogether. 

The focus of this experiment was to determine the quality of bugs, and Mythril appears to surpass both tools in this area. The bugs found by Oyente were primarily related to tasks such as integer overflow, and the primary bugs found by SmartCheck were \texttt{SOLIDITY\_VISIBILITY} bugs where a variable or function is public when it can be private. The Mythril suite found bugs associated to bad parameters that could be used as back doors for attacks, such as passing a raw address to a contract. Thus, although in many cases both Oyente and SmartCheck found more bugs, the Mythril suite finds better bugs.


